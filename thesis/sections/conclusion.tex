\chapter{Discussion}
\label{sec:conclusion}


% Eerst conclusie, dan discussie, dan future work
% TODO: Discussie: Wat had er beter gekund aan het onderzoek?
% TODO: Meer future work (wat had ik nog allemaal willen doen?)
% TODO: Vragen hoe Sander vindt dat dit hoofdstuk moet heten

% Conclusion:
\section{Conclusion}
We can conclude that the Option MCTS algorithm almost always performs at least
as good as MCTS. O-MCTS excels in games with a small level grid or little
sprites, but high complexity, such as \textit{zelda}, \textit{overload} and
\textit{eggomania}.  Furthermore, O-MCTS can do a further lookahead than most
tree searching alternatives, resulting in a high performance on games like
\textit{camel race}. The algorithm performs worse than expected in games with a
high amount of sprites, since the size of the option set becomes so large, that
maintaining it takes a lot of time, leaving too little time for exploring
options. 

The results of OL-MCTS indicate that it is possible to learn some information
about which options work better, meaning that in the future, it should be
possible to completely remove infeasible options from the option set, that have
low expected rewards for certain games. We expect that this could reduce the
computation time O-MCTS needs to construct and check all the options. However,
more work needs to be done in this area to enable more improvement.

\section{Discussion}
This section critically discusses the research that is shown in this thesis and
assumptions that have been made alongside it. Firstly, the trade-offs between
scoring, winning and playing time are discussed. Then, the evaluation of the
SMDP Q-learning algorithm is discussed and lastly, this section discusses why
OL-MCTS does not introduce a significant improvement upon O-MCTS.

% Trade off between score and winning
A difficulty in game solving research is the trade-off between scoring, playing
time and winning. First the trade off between playing time and winning: winning
earlier, is often better. For some games, however, winning later increases the
score. For example, in the game \textit{butterflies}, butterflies keep spawning
indefinitely. Each butterfly that is caught, increases the score.  When all the
butterflies are gone, the game is won. An algorithm that catches all the
butterflies as quickly as possible, will have a lower score than an algorithm
that catches them less efficiently. The question to ask then is: which of these
is better? The one with a high score, or the one that finishes quickly.

Another trade-off is between winning and scoring: in the game Zelda, for
instance, the goal is to pick up the key and walk to the portal. If the monsters
are killed, this increases the eventual score. But trying to kill a monster
comes at a risk: the player has to come close to the monster in order to use his
sword on it. So increasing the score, comes with the risk of dying, decreasing
the win ratio. It is unclear how these trade-offs should be handled.

% Q-learning should get more time to load its table
In section \ref{subsec:experiments-smdp-qlearning}, SMDP Q-learning was allowed
one second of initialization time, as per the rules of the GVGAI competition.
The comparison in this thesis, between SMDP Q-learning and OL-MCTS, is fair
because they both have to adhere the GVGAI competition rules. It would, however,
be interesting to do more experiments with a longer initialization time, because
Q-learning could learn for more games without getting disqualified and its
performance might improve.

% Smaller option set
The experiments that have been done in Chapter \ref{sec:experiments}, all use
the same option set. More experiments need to be done to find out the influence
of each of the options. The addition of a bad option in the option set might add
overhead without improving the mean score of the algorithm on any of the games.
Bad options should be taken out of the option set, improving the performance of
O-MCTS by removing overhead. 

The OL-MCTS algorithm was constructed to reduce the overhead introduced by the
O-MCTS algorithm, but offers little improvement on most games. The current
hypothesis is that because the algorithm does not completely exclude the poorly
performing options from the option set, little of the overhead is reduced. It
should be possible to set a threshold to the option values, below which the
options are not maintained. This should reduce the overhead of the algorithm,
but could introduce overfitting: options might be excluded prematurely, e.g.,
the \texttt{GoToPosition} option for the portal in Zelda, that is closed at
first, should not be discarded because after picking up the key it becomes
useful. 

A similar limitation of the OL-MCTS algorithm, is that it does not learn the
conditions in which options are good: it only learns the mean and variance of an
option's return. Although the crazy stone algorithm is inclined to try options
with a high variance, an algorithm that can learn the conditions in which these
options result in a score improvement can yield better results.

\section{Future Work}
This section discusses future research that can be conducted to improve O-MCTS
and OL-MCTS and to get a better understanding of the algorithms. 

Firstly, more research should be done in the individual options. Currently the
effectiveness of the algorithm relies on the A Star implementation.  This
implementation is too time consuming, leaving little computation time for tree
building. In future work, trials should be done with simpler and computationally
cheaper alternatives, such as for example Enforced Hill Climbing, as proposed in
\cite{ross2014general}, although that has the problem that the agent can get
stuck. Alternatively, by creating goal-oriented MDPs similar to PUMA's, the
algorithm could probably increase sturdiness, although this will be hard in the
context of the GVGAI competition.

% Remove options from the option set after n games have been won
Furthermore, OL-MCTS might be improved by removing options from the option set
$O$ after several games have been won. By requiring the algorithm to first win
some of the games, it is possible to know which options never contribute to a
positive score change. These options can be removed from the option set,
reducing the time it takes to maintain it.

% No/periodical interruption
Early in the development process, many of the options resulted in the agent
getting killed because it was following a specific option for too long. For this
reason, as described in Section \ref{subsec:options}, interruption was
introduced. In the mean time, the sturdiness of the options increased, but
because interruption was embedded in the algorithms, it was never disabled.
Although we hypothesize that this contributes to the performance of the
algorithm, research needs to be done in the impact it has on the algorithms.

% Combining calculation time of several actions, (like Purofvio)
If interruption is disabled, the computation time of several actions can be
used to build the tree for the next option. This means that whereas normally a
tree has to be built in the time for one action, with interruption disabled, it
can be built in the action time of the duration of an option. This multiplies
the available computation time and is expected to have a positive effect on the
performance of O(L)-MCTS.

% PTSP (like Purofvio)
In Chapter \ref{sec:background}, the Purofvio algorithm is descibed. This
algorithm is comparable to O-MCTS, but was only tested on the physical traveling
salesperson problem. O-MCTS should be run on the PTSP problem as well, in order
to compare these algorithms.

% Test if there really is no preference for longer options
Powley et al. suggest that Purofvio always prefers longer options over shorter
ones. We hypothesize that O-MCTS does have this problem, due to the discounting
that is done by the backup function.  This has to be tested by creating an
option set with two options.  One that achieves a subgoal and one that achieves
the same goal with more actions. If O-MCTS prefers the shorter one, we can
conclude that our backup method indeed solves Purofvio's problem.

% Multi-objective?
Multi-objective algorithms do not optimize one objective, but search for 
optimal solutions that maximize a combination of several objectives at once
instead. A multi-objective variant of MCTS exists, on which a multi-objective
variant of O-MCTS can be based. Using a multi-objective variant of the
algorithm, it might be possible to balance the trade-offs between playing time,
score and winning discussed in the previous section. 

% Other option value methods
Lastly, in order to improve the learning algorithms, some other improvements can be
investigated. Firstly, the backup method can be tweaked. In
\cite{coulom2007efficient}, instead of the mean value, other values, like the
maximum return are used in the backup phase. Furthermore, the mean and standard
deviation of option returns are now calculated over all the games, without
regarding how long ago this game was played. This might lead to underrated
options, for example with doors that unlock under specific conditions (for
example when the key is picked up in \textit{zelda}).  Using a maximum return,
or discounting the option values might have a different effect.

