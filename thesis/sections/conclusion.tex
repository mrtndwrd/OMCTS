\chapter{Discussion}
\label{sec:conclusion}


% Eerst conclusie, dan discussie, dan future work
% TODO: Discussie: Wat had er beter gekund aan het onderzoek?
% TODO: Meer future work (wat had ik nog allemaal willen doen?)
% TODO: Vragen hoe Sander vindt dat dit hoofdstuk moet heten


We can conclude that the Option MCTS algorithm almost always performs at least
as good as MCTS. O-MCTS excels in games with a small amount of game sprites, but
high complexity, such as \textit{zelda}, \textit{overload} and \textit{eggomania}.
Furthermore, O-MCTS can do a further lookahead than most tree searching
alternatives, resulting in a high performance on \textit{camel race}. The
algorithm does struggle with a high amount of sprites, since many options have
to be constructed and checked each round, leaving too little time for exploring
options. Our results of OL-MCTS indicate that learning
option values can aid the algorithm in improving win ratio and score, but not
for all games.

Our results with OL-MCTS indicate that it is possible to detect infeasible
options, meaning that in the future, it should be possible to completely remove
options from the option set, that have low expected rewards for certain games.
We expect that this could reduce the computation time O-MCTS needs to construct
and check all the options.

Furthermore, more research should be done in the influence of the option set.
Currently the effectiveness of the algorithm strongly relies on the A Star
implementation. This implementation is too time consuming, leaving
little time for actual tree building. In future work, trials can be done with
simpler and computationally cheaper alternatives, such as for example Enforced
Hill Climbing, as proposed in \cite{ross2014general}, although that has the
problem that the agent can get stuck. Alternatively, by creating goal-oriented
MDPs similar to PUMA's, the algorithm could probably increase sturdiness.

In order to improve the learning algorithms, some other improvements can be
investigated. Firstly, the backup method can be tweaked. In
\cite{coulom2007efficient}, in stead of the mean value, other values, like the
maximum return are used in the backup phase. Furthermore, the mean and standard
deviation of option returns are now calculated over all the games, without
regarding how long ago this game was played. This might lead to underrated
options, for example with doors that unlock under specific conditions (for
example when the key is picked up in \textit{zelda}).  Using a maximum return,
or discounting the option values might have a different effect.

