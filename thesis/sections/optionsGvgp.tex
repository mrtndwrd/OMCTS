\chapter{Options for General Video Game Playing}

Options have previously mostly been used in specific cases and for relatively
simple problems. To specify the problem domain, this chapter will cover how a
game is defined in VGDL and how it is observed by the game playing algorithm.
Furthermore, we will explain what options mean in the context of general video
game playing and what we have done to create a set of options that can be used
in any game. Lastly, this chapter explains what is needed to use SMDP Q-learning
on the domain of general video game playing.

\section{Toy Problem}

As described in the background section, the foundation of a game lies in two
specifications. The game dynamics, and levels. The game dynamics are typically
the same for each level and do not change over the course of a game. Levels
define the layout of the game. Each level is different and typically the last
level of a game is harder than the first level.

In VGDL, the game description defines the game dynamics. Each game has one game
description file, which describes the game sprites and their interaction with
each other. The levels are defined in separate level files and define the layout
of the screen.

In this section, we introduce our test game \emph{prey}. We created this game
for testing the learning capacities of the algorithm. Furthermore, this section
contains an in-depth explanation of the VGDL. The game aims to be simple to
understand and easy to win. It should be possible to see any improvement an
algorithm could achieve as well. We chose the \emph{predator \& prey} game, in
which the player is a predator, that should catch its prey (an NPC) by walking
into it. We decided to have three types of prey, one that never moves, one that
moves once in 10 turns and one that moves once in 2 turns. This section
describes how the game is made in VGDL and how a GVGAI agent can interact with
it.

\lstinputlisting[basicstyle=\footnotesize, frame=tb, xleftmargin=.1\textwidth, %
xrightmargin=.1\textwidth, caption=prey.txt, label=lst:preytxt, float=t]%
{../examples/gridphysics/prey.txt}

\begin{minipage}[t]{.4\textwidth}
	\lstset{
		caption=Prey level 1, 
		label=lst:prey1,
		basicstyle=\footnotesize, frame=tb,
		xleftmargin=.1\textwidth, xrightmargin=.1\textwidth
	}
	\begin{lstlisting}
wwwwwww
wA    w
w     w
w     w
w    Iw
wwwwwww
	\end{lstlisting}
\end{minipage}
\begin{minipage}[t]{.5\textwidth}
	\lstset{
		caption=Prey level 2,
		label=lst:prey2,
		basicstyle=\footnotesize, frame=tb,
		xleftmargin=.1\textwidth, xrightmargin=.1\textwidth
	}
	\begin{lstlisting}
wwwwwwwwwwwww
wA     w    w
w      w    w
w      w    w
w wwwwww    w
w           w
w           w
w           w
w           w
w           w
w           w
w       wwwww
w          Iw
wwwwwwwwwwwww
	\end{lstlisting}
\end{minipage}

The code in Listing \ref{lst:preytxt} contains the game description. Lines 2 to
8 describe the available sprites. There are two sprites in the game, which are
both \texttt{movable}: the avatar (predator) and the monster (prey).  The avatar
is of type \texttt{MovingAvatar}, which means that the player has four possible
actions (up, right, down, left). The prey has three instantiations, all of the
type \texttt{RandomNPC}, which is an NPC that moves about in random directions:
the \texttt{inactivePrey}, which only moves every 3000 steps (which is more than
the timeout explained shortly, so it never moves); the \texttt{slowPrey} which
moves once every 10 steps, and the \texttt{fastPrey} which moves once every 2
steps.  By default, the \texttt{MovingAvatar} can move once in each time step.

Lines 10 to 14 describe the level mapping. These characters can be used in the
level description files, to show where the sprites spawn. 

In the interaction set, Line 17 means that if the prey walks into the avatar (or
vice versa) this will kill the prey, and the player will get a score increase of
one point. Line 18 dictates that no \texttt{movable} sprite can walk through
walls.

Lastly, in the termination set, line 21 shows that the player wins when there
are no more sprites of the type \texttt{prey}, and line 22 shows that the player
loses after 100 time steps.

Listing \ref{lst:prey1} shows a very simple level description. This level is
surrounded with walls and contains one avatar and one inactive prey. 
The game can be used to test the functionality of an algorithm. The first level
is very simple and is only lost when the agent is not able to find the prey
within the time limit of 100 time steps, which is unlikely. A learning
algorithm should, however be able to improve the number of time steps it needs
to find the prey. The minimum number of time steps to win the game in the first
level, taking the optimal route, is six. The second level, shown in Listing
\ref{lst:prey2}, is more complex. The agent has to plan a route around the
walls and the prey is further away. We chose to still use the inactive prey,
because then we know the exact minimum number of time steps needed to win, which
in this case is twenty.

\todo{observation}


\section{Option Set}
\todo{- Wat zijn options die mensen bedenken bij het zien van een game?}


\todo{Zeggen dat dit de concrete implementatie van options is}
For these experiments we construct an option set which is aimed at providing
action sequences for any type of game, since the aim here is general video game
playing. Note that a more specific set of options can be created when the
algorithm should be tailored to only one type of games and similarly, more
options can be added to the algorithm easily.

\begin{itemize}[noitemsep]
	\item \texttt{ActionOption} executes a specific action once and then
		stops.
	\item \texttt{AvoidNearestNpcOption} makes the agent avoid the nearest NPC
	\item \texttt{GoNearMovableOption} makes the agent walk towards a
		movable game sprite (defined as movable by the VGDL) and stops when it
		is within a certain range of the movable
	\item \texttt{GoToMovableOption} makes the agent walk towards a
		movable until its location is the same as that of the movable
	\item \texttt{GoToNearestSpriteOfType} makes the agent walk to the nearest sprite of
		a specific type
	\item \texttt{GoToPositionOption} makes the agent walk to a specific position
	\item \texttt{WaitAndShootOption} waits until an NPC is in a specific location and
		then uses its weapon.
\end{itemize}

For each option type, a subtype per visible sprite type is created during the
game. For each sprite, an option instance of its corresponding subtype is
created. For example, the game \textit{zelda}, as seen in Figure \ref{fig:zelda},
contains three different sprite types (excluding the avatar and walls);
monsters, a key and a portal. The first level contains three monsters, one key
and one portal, and the aim of the game is to collect the key and walk towards
the portal without walking into the monsters. The score is increased by 1 if a
monster is killed (i.e., its sprite is on the same location as the sword sprite)
if the key is picked up, and when the game is won. \texttt{GoToMovableOption} and
\texttt{GoNearMovableOptions} are created for each of the three monsters and
for the key. A \texttt{GoToPositionOption} is created for the portal.  One
\texttt{GoToNearestSpriteOfType} is created per sprite type. One
\texttt{WaitAndShootOption} is created for the monsters, and one
\texttt{AvoidNearestNpcOption} is created. This set of options is $O$ in
Algorithms \ref{alg:omcts} and \ref{alg:olmcts}. In a state where, for example,
all the monsters are dead, the possible option set $\mathbf{p}_s$ does not
contain the \texttt{AvoidNearestNpcOption} and \texttt{GoToMovableOption}s and
\texttt{GoNearMovableOption}s for the monsters.

The \texttt{GoTo\ldots} options all utilize an adaptation of the A Star
algorithm to plan routes. An adaptation is needed, because at the beginning of
the game there is no knowledge of which sprites are traversable and which are
not. Thus, during every move that is simulated by the agent, the A Star module
has to update its beliefs about the location of walls and other blocking
objects. This is accomplished by comparing the movement the avatar wanted to
make to the movement that was actually made in game. If the avatar did not move,
it is assumed that all the sprites on the location the avatar should have
arrived in are blocking sprites. A Star keeps a \emph{wall score} for each
sprite type. When a sprite blocks the avatar, its wall score is increased by
one. Additionally, when a sprite kills the avatar, its wall score is increased
by 100, in order to prevent the avatar from walking into killing sprites.
Traditionally the A Star's heuristic uses the distance between two points. Our A
Star adaptation adds the wall score of the goal location to this heuristic,
encouraging the algorithm to take paths with a lower wall score. This method
enables A Star to try to traverse paths that were unavailable earlier, while
preferring safe and easily traversable paths. For example in \textit{zelda}, a
door is closed until a key is picked up. Our A Star version will still be able
to plan a path to the door once the key is picked up, winning the game.

The GVGAI competition comes with some predefined parameters. Every algorithm has
a maximum of one second to initialize, before starting the game. Then, the
algorithm has a maximum forty milliseconds to choose an action, after which a
new game state is given. If the agent takes more time than that, the action
\textsc{null} will be applied and the avatar will not do anything\footnote{There
seems to be a bug in the system, that causes actions to sometimes take hundreds
of milliseconds to simulate. This bug has been reported to the competition's
discussion platform by several users. Because this bug was not solved at the
time of writing, we chose to not honor the GVGAI competition's original rule to
disqualify a controller that does not return an action within fifty
milliseconds.}. The framework was edited in order to be able to pass information
from one game to the next. After a game is finished, the learning algorithms
OL-MCTS and Q-learning get the possibility to save their values to a file. Our
restriction is that the file can be opened during the initialization of the next
game, within one second. 

We empirically optimize the parameters of the O(L)-MCTS algorithm for these
experiments. We use discount factor $\gamma = 0.9$. The maximum search depth $d$
is set to 70, which is higher than most alternative tree search algorithms, for
example in the GVGAI competition, use. The number of node visits after which
\textsf{uct} is used, $v$, is set to 40. Crazy stone parameter $K$ is set to
$0.5$.  For comparison, we use the MCTS algorithm provided with the Java
implementation of VGDL with its default value of maximum search depth of 10.
Both algorithms have \textsf{uct} constant $C_p = \sqrt{2}$. Unfortunately,
comparing to Q-learning with options was impossible, because the state space of
these games is too big for the algorithm to learn any reasonable behavior. All
the experiments are run on an Intel %\textsuperscript{\textregistered} 
Core %\texttrademark 
i7-2600, 3.40GHz quad core processor with 6 GB of DDR3, 1333 MHz RAM memory. For
all experiments, each algorithm plays each of the 5 levels of each game 20
times. 

\section{SMDP Q-learning for General Video Game Playing}
\todo{- Laatste paragraaf: SMDP Q-learning for GVGP%
	- Wat hebben we aan aanpassingen gedaan%
	- Als we deze options gebruiken voor SMDP Q-learning, leidt dat tot een%
	algoritme, dat wordt experimenteel getest in section 7%
	- Beschrijven voor- en nadelen van SMDP Q-learning toegepast op games %
		- Deze disadvantages worden ge-metegate door ons nieuwe algoritme}
