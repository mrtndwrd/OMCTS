\chapter{OL-MCTS: Learning Option Values} 
\label{sec:learning} 

Although we expect O-MCTS is an improvement over MCTS, we also expect the
branching factor of O-MCTS's search tree to increase as the number of options
increases.  When many options are defined, exploring all the options becomes
infeasible. In this chapter, we will define \emph{option values}: the expected
mean and variance of an option. These can be used to estimate which options need
to be explored, and which do not. We adjust O-MCTS to learn the option values
and focus more on the options that seem more feasible. We call the new algorithm
\emph{Option Learning MCTS (OL-MCTS)}. We expect that OL-MCTS can create deeper
search trees than O-MCTS in the same amount of time, which results in more
accurate node values and an increased performance. Furthermore, we expect that
this effect is the greatest in games where the set of possible options is large,
or where only a small subset of the option set is needed in order to win.

In general, OL-MCTS saves the return of each option after it is finished, which
is then used to calculate global option values. During the expansion phase of
OL-MCTS, options that have a higher mean or variance in return are prioritized.
Contrary to O-MCTS not all options are expanded, but only those with a high
variance or mean return. The information learned in a game can be transferred if
the same game is played again by supplying OL-MCTS with the option values of the
previous game.

The algorithm learns the option values, $\mu$ and $\sigma$. The expected mean
return of an option $o$ is denoted by $\mu_o$. This number represents the
returns that were achieved in the past by an option for a game. It is
state-independent. Similarly, the variance of all the returns of an option $o$
is saved to $\sigma_o$.

For the purpose of generalizing, we divide the set of options into \emph{types}
and \emph{subtypes}. The option for going to a movable sprite has type
\texttt{GoToMovableOption}. An instance of this option exists for each movable
sprite in the game. A subtype is made for each sprite type (i.e., each
different looking sprite). The option values are saved and calculated per
subtype. Each time an option $o$ is finished, its subtype's values $\mu_o$ and
$\sigma_o$ are updated by respectively taking the mean and variance of all the
returns of this subtype. The algorithm can generalize over sprites of the same
type by saving values per subtype.

Using the option values, we can incorporate the progressive widening algorithm,
crazy stone, from Equation \ref{eq:crazystone} to shift the focus of exploration
to promising regions of the tree. The crazy stone algorithm is applied in the
expansion phase of OL-MCTS.  As a result, not all children of a node will be
expanded, but only the ones selected based on crazy stone. When using crazy
stone, we can select the same option several times, this enables deeper
exploration of promising subtrees, even during the expansion phase. After a
predefined number of visits $v$ to a node, the selection strategy \textsf{uct}
is followed in that node to tweak the option selection. When it starts using
\textsf{uct}, no new expansions will be done in this node.

\begin{algorithm}[h]
	\caption{$\mathsf{OL-MCTS}(O, r, t, d, v, \mu, \sigma)$}
	\label{alg:olmcts}
	\begin{algorithmic}[1]
		\State $C_{s \in S} \gets \emptyset$
		\State $\mathbf{o} \gets \emptyset$
		\While {$time\_taken < t$} \label{alg:olmcts:mainloop}
			\State $s \gets r$
			\While {$\neg \mathsf{stop}(s, d)$} \label{alg:olmcts:innerloop}
				\If{$s \in \beta(o_s)$} \label{alg:olmcts:sp}
					\State $\mathsf{update\_values}(s, o_s, \mu, \sigma)$
						\Comment{update $\mu$ and $\sigma$} \label{alg:olmcts:update}
					\State $\mathbf{p}_s \gets \cup_o (s \in I_{o \in O})$
				\Else
					\State $\mathbf{p}_s \gets \{o_s\}$
				\EndIf \label{alg:olmcts:scs}
				\State $\mathbf{m} \gets \cup_o (o_{s \in \mathbf{c}_s})$
				\If{$n_s < v$} \Comment{if state is visited less than $v$ times}
					\label{alg:olmcts:ns}
					\State $\mathbf{u}_s \gets \mathsf{crazy\_stone}(\mu, \sigma, \mathbf{p}_s)$
						\Comment{apply crazy stone, Eq. \ref{eq:crazystone}}
					\State $\omega \gets \mathsf{weighted\_random}(\mathbf{u}_s, \mathbf{p}_s)$
					\If{$\omega \not\in \mathbf{m}$} \Comment{option $\omega$ not expanded}
						\State $a \gets \mathsf{get\_action}(\omega, s)$ \label{alg:olmcts:scs}
						\State $s' \gets \mathsf{expand}(s, a)$ 
						\State $\mathbf{c}_s \gets \mathbf{c}_s \cup \{s'\}$
						\State $o_{s'} \gets \omega$
						\State \textbf{break} \label{alg:olmcts:ecs}
					\Else \Comment{option $\omega$ already expanded}
						\State $s' \gets s \in \mathbf{c}_s : o_s = \omega$ \label{alg:olmcts:s} \Comment{select child node that uses $\omega$}
					\EndIf
				\Else \Comment{apply \textsf{uct}}
					\State $s' \gets \mathsf{uct}(s)$ \label{alg:olmcts:uct}
				\EndIf \label{alg:olmcts:ecs}
				\State $s \gets s'$ \label{alg:olmcts:ss}
			\EndWhile
			\State $\delta \gets \mathsf{rollout}(s')$ \label{alg:olmcts:rollout}
			\State $\mathsf{back\_up}(s', \delta)$ \label{alg:olmcts:backup}
		\EndWhile
	\end{algorithmic}
\end{algorithm}

The new algorithm can be seen in Algorithm \ref{alg:olmcts} and has two major
modifications. The updates of the option values are done in line
\ref{alg:olmcts:update}. The function \textsf{update\_values} takes the return
of the option $o$ and updates its mean $\mu_o$ and variance $\sigma_o$ by
calculating the new mean and variance of all returns of that option subtype. The
second modification starts on line \ref{alg:olmcts:ns}, where the algorithm
applies crazy stone if the current node has been visited less than $v$ times, or
alternatively applies UCT similarly to O-MCTS. The \textsf{crazy\_stone}
function returns a set of weights over the set of possible options
$\mathbf{p}_s$. A weighted random then chooses a new option $\omega$ by using
these weights.  If $\omega$ has not been explored yet, i.e., there is no child
node of $s$ in $c_s$ that uses this option, the algorithm chooses and applies an
action and breaks to rollout in lines \ref{alg:olmcts:scs} to
\ref{alg:olmcts:ecs}. This is similar to the expansion steps in O-MCTS. If
$\omega$ has been explored in this node before the corresponding child node
$s'$ is selected from $c_s$ and the loop continues like when \textsf{uct}
selects a child.

We expect that by learning option values and applying crazy stone, the algorithm
can create deeper search trees than O-MCTS. These trees are focused more on
promising areas of the search space, resulting in improved performance.
Furthermore, we expect that by transferring option values to the next game, the
algorithm can improve after replaying games.
