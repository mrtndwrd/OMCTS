\chapter{Introduction}
\label{sec:introduction}

In the history of decision theory, playing games has been one of the objectives
for a long time. The algorithms typically maximize their score or win
probability. Early AI algorithms focused on simple games like tic tac toe.
Later focus was shifted to chess and even later to Go. Nowadays, many algorithms
are designed for winning computer games, for example a lot of strategy games
offer the player computer-controlled contestants. 
An increasingly important goal in decision theory is creating AI algorithms that
are capable of solving more than one problem. A general AI, instead of a
specific alternative, is considered a step towards creating a strong AI.

An especially challenging domain of AI is general video game playing.  Since
many real-world problems can be modeled as a game, algorithms that can play
complex games successfully are often highly effective problem solvers as well.
Therefore, recent research focuses on algorithms capable of solving several
games with differing types of objectives. A common general game solving approach
is to use a tree search in order to select the best action for any given game
state. In every new game state, the tree search is restarted, until the game
ends. 

A popular decision theoretic algorithm, that has been used in general video game
playing as well, is \emph{Monte Carlo tree search (MCTS)}. Decision theory and
general video game playing are different research areas, but both can benefit
from each other. For general video game playing, an increase of performance can
be found using algorithms designed for complex decision-theoretic problems,
whereas the algorithms that are created for general video game playing can be
applied to other decision-theoretic problems.  Furthermore, applying decision
theoretic algorithms to a new class of problems, like general video game
playing, can lead to a better understanding of the algorithms, and new insights
in their strengths and weaknesses.  MCTS owes its fame to playing games
\cite{gelly2006modification}, but is used for other decision-theoretic problems
as well, e.g., scheduling problems or combinatorial optimization problems (see
\cite{browne2012survey}, section 7.8 for an extensive list). 

A method to test the performance of a general video game playing algorithm is
by using the framework of the \emph{general video game AI (GVGAI)} competition
\cite{perez2014}.  A number of algorithms have been submitted to this competition,
of which many rely on a tree search method. For instance, there are submissions
based on an open loop tree search \cite{perez2015open}, these algorithms
typically evaluate a game state more than once, in contrast to a closed loop
tree search (like the standard MCTS algorithm) where each action is only
simulated once during tree building. 

A limitation in many tree search algorithms, is that many games are too complex
to plan far ahead in a reasonable time frame. Planning by tree search based
methods often only considers short-term score differences and does not
incorporate long-term plans. Moreover, many algorithms lack common video game
knowledge and do not use any of the knowledge gained from the previous games.

In contrast, when humans play a game we expect them to do assumptions about its
mechanics, e.g., pressing the left button often results in the player's avatar
moving to the left on the screen. Players can use these assumptions to learn how
to play a game more quickly. Furthermore, human players have an abstraction
layer over their action choices; instead of choosing one action at a time they
define a specific subgoal for themselves: when there is a portal
on screen, a human player is likely to try to find out what the portal does by
walking towards its sprite (the image of the portal on the screen). The player
will remember the effect of the portal sprite, and use that information for the
rest of the game. In this case, walking towards the portal can be seen as a
subgoal of playing the game.

In certain situations, it is clear how such a subgoal can be achieved and a
sequence of actions, or \emph{policy}, can be defined to achieve it. A policy to
achieve a specific subgoal is called an option. Thus, an option selects an
action, given a game state, that aims at satisfying its subgoal. Options, in
this context, are game-independent. For example, an option that has reaching a
specific location in the game (for example a portal) as its objective, selects
actions using a path planning heuristic that will reach the goal location. 
The probability of this kind of subgoal being achieved by an algorithm that
does not use options is smaller, especially when the road to the subgoal does
not indicate any advantage for the player. For instance, In the first few
iterations of MCTS, the algorithm will be equally motivated to move 10 steps
into the direction of a certain game sprite as it will be to do any other
combination of 10 actions. 

An existing option planning and learning approach is \emph{SMDP Q-learning}. It
was originally proposed for solving \emph{semi-Markov decision processes
(SMDPs)} which are problems with a continuous action time
\cite{duff1995reinforcement} and later used in combination with
options to navigate a grid world environment \cite{sutton1999between,
stolle2002learning}.  
The traditional Q-learning is adapted to apply the update rules for SMDPs to
problems with a given set of options, in order to be able to find the optimal
option for each game state. This thesis will explain how SMDP Q-learning can be
implemented for general video game solving in the GVGAI competition.  

However, SMDP Q-learning does not have the same favorable properties as Monte Carlo tree
search. For instance, although they are both anytime algorithms (they can both
return an action that maximizes their hypothesis at any time), SMDP Q-learning
usually has to play a game several times, before it can return good actions,
whereas MCTS can return reasonable actions with less simulations.  We expect
that combining MCTS with the option framework yields better results.

Therefore, we propose a new algorithm called \emph{option Monte Carlo tree
search (O-MCTS)} that extends MCTS to use options. Because O-MCTS chooses
between options rather than actions when playing a game, we expect it to be able
to plan at a higher level of abstraction. Furthermore, we introduce \emph{option
learning MCTS (OL-MCTS)}, an extension of O-MCTS that approximates which of the
options in the option set is more feasible for the game it is playing. This can
be used to shift the focus of the tree search exploration to more promising
options. This information can be transferred in order to increase performance on
the next level.

Our hypothesis is that by using options, exploration can be guided to find out
the function of specific game objects, leading to better results while playing
the game. In this thesis, we aim to incorporate the use of options in general
video game playing, hypothesising that the use of options speeds up planning in
the limited time algorithms often have before taking a decision. 

The new algorithms are benchmarked on games from the General Video Game AI
competition, against SMDP Q-learning and the Monte Carlo tree search algorithm
that is provided by that competition. For these experiments, a specific set of
options has been constructed which aims to provide basic strategies for game
playing, such as walking towards or avoiding game sprites and using a ranged
weapon in a specific manner. Our results indicate that the O-MCTS and OL-MCTS
algorithms outperform traditional MCTS in games that require a high level of
action planning, e.g., games in which something has to be picked up before a
door can be opened. In most other games, O-MCTS and OL-MCTS perform at least as
good as MCTS.
