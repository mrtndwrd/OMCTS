\chapter{Related Work}
\label{sec:related}
This chapter covers some popular alternative methods for general video game
playing and prior work on tree search with options. \emph{Deep Q networks (DQN)}
is a popular algorithm that trains a convolutional neural network for a game
\cite{mnih2013playing}, \emph{Planning under Uncertainty with Macro-Actions
(PUMA)} is a forward search algorithm that uses extended actions in
\emph{partially observable MDPs (POMDPs)} \cite{he2010puma}.  \emph{Purofvio} is
an algorithm that combines MCTS with macro-actions that consist of repeating one
action several times \cite{powley2012monte}. 

DQN is a general video game playing algorithm that trains a convolutional neural
network that has the last four pixel frames of a game as input and tries to
predict the return of each action. A good policy can then be created by
selecting the action with the highest return. This is one of the first
algorithms that successfully combines neural network learning with reinforcement
learning. In this case, however, it was not desirable to implement DQN, because
in our testing framework, an algorithm has 40 milliseconds to choose an action,
even when playing a game for the first time. DQN has to train for several days,
before a good policy can be found.  Furthermore, it does not use a forward
model, but always applies actions directly to the game in order to learn.

Although learning is currently not a goal in the GVGAI competition and not
supported by the rules, there are results showing that several types of learning
algorithms can improve their scores on many game plays, using a simple state
representation \cite{samothrakis2015neuroevolution}. Their features consist
solely of: 
\begin{itemize}[noitemsep]
	\item the game score, 
	\item the game tick,
	\item the winner ($-1$ if game is still ongoing, $0$ if the player lost and
		$1$ if the player won), 
	\item game over ($0$ if the game is ongoing, $1$ if the game is over), 
	\item a list of resources, 
	\item a list of Euclidean distances to the nearest sprite of each type,
	\item the speed of the avatar. 
\end{itemize}
The results of the paper show that the algorithms are capable of learning in the
course of 1000 game plays of the first level of each game. It has to be noted
that no results of how many times the algorithms \emph{win} the game are
reported and that it seems (looking at the score that is achieved) that many of
the games are actually lost most of the times. Furthermore, the learning
algorithms proposed in this thesis will focus more on early results than on long
term learning.

An alternative tree search algorithm is PUMA, which applies forward search to
options (referred to as macro-actions) and works on POMDPs.  PUMA automatically
generates goal-oriented MDPs for specific subgoals, the advantage of which is
that using this method, effective options can be generated without requiring any
prior knowledge of the (PO)MDP. The disadvantage is that this takes a lot of
computation time and thus would not work in the GVGAI framework, where only 40
milliseconds of computation time is allowed between actions. Options generated
for one game, would not necessarily be transferable to other games, meaning that
option generation would have to be done prior to every game that the algorithm
plays. Furthermore PUMA has to find out the optimal length per macro-action,
whereas the algorithms proposed in this thesis can use options of variable
length with starting an stopping conditions.

Another algorithm that uses MCTS with macro actions is called Purofvio.
Purofvio plans over macro-actions which, in this case, are defined as repeating
one action for a fixed number of times. No more complex options are defined.
Generally, when executing an option, Purofvio executes only the actions chosen
by that option, reducing the branching of MCTS. The algorithm is constructed for
the physical traveling salesperson problem, in which (like GVGAI) the action
time is set to 40 ms. An important feature of this algorithm is that it can use
the time budget of several actions to compute which macro-action to choose next.
This is possible, because when a macro-action has been chosen, it is not
reconsidered. The paper notes that their options must always be of the same
size, because they found that otherwise MCTS seems to favor options with a
longer time span over shorter options. It is suggested that Purofvio could work
on other games as well, but this has not been shown.
