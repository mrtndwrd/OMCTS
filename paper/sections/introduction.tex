\section{Introduction}
\label{sec:introduction}
%\todo{Wat is AI in games?}
%\todo{Er bestaat game-specifieke AI, die werkt goed}
In the history of AI algorithms, solving games has been one of the objectives
for a long time. These AI algorithms typically maximize their score or win
probability. One of the first AI algorithms was focused on solving tic tac
toe. Later, focus was shifted to chess and even later to Go. Nowadays, many
algorithms are designed for solving computer games. For example a lot of
strategy games offer computer controlled contestants for the player to
challenge. AI in computer games is mostly designed with a specific game in mind.
\todo{examples of good working algorithms?} 

%	\todo{Volgende stap: General AI}
%		\todo{Bijv. GVGAI competitie}
These game-specific AIs mostly work very well, but it is a very time consuming
task to create an AI for every possible game. Therefore recent research focusses
on algorithms capable of solving several games with different types of
objectives. Algorithms for general game solving can be compared to each other in
the \emph{General Video Game AI} competition\cite{perez2014}. 

%\todo{Als mensen games spelen, maken ze gebruik van hun algemene kennis over
%games.}
\todo{This is kinda lonely here. Move it to a logical location}
Humans playing a game for the first time use knowledge and skills they
obtained from other games in order to improve their performance. A good
general video game algorithm should also have this feat. 

%\todo{Meest effectieve algoritmes op dit moment zijn gebasseerd op MCTS, maar doen
%nog weinig met algemene kennis, daarom onze bijdrage.}
Most of the current competitors in the GVGAI competition use a tree-based search
in order to select the best action to take in a state. This tree search is 
repeated in every state until the game ends. Most of the algorithms lack
common video game knowledge and don't use any of the knowledge gained over the
past games (in fact, the GVGAI framework by default prohibits algorithms to
transfer any information to the next game).

%\todo{Introductie van options (abstracter denkniveau)}
%	\todo{Introductie van leren over options}
%	\todo{Options zijn nog nooit gebruikt in MCTS, die combinatie is een nieuwe
%	toevoeging.}
In this paper a new algorithm is introduced that uses \emph{options}, which can be seen
as a form of prior knowledge. Options are sequences of actions with a specific
goal, for example going somewhere or avoiding a certain sprite of the game.
Over the course of several games information about which options are helpful and
which are not can be learned, improving the overall performance of the
algorithm. This information can also be transfered in order to increase
performance on a next, possibly harder, level.
