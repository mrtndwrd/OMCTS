\section{Experiments}
\label{sec:experiments}
\begin{figure*}
\centering
% \subfigure[wins]{%
% \includegraphics[width=\textwidth]{includes/wins}%
% }
% %\vspace{-.4cm}
% %\caption{Win ratio of the algorithms per game on all levels.}
% %\label{fig:wins}
% %\centering
% \subfigure[scores]{%
% \includegraphics[width=\textwidth]{includes/scores}%
% }
	\includegraphics[width=\textwidth]{includes/wins}
	\includegraphics[width=\textwidth]{includes/scores}
	\vspace{-.8cm}
\caption{Win ratio and mean normalized score of the algorithms per game. O-MCTS
outperforms MCTS.}
\label{fig:scores}
\end{figure*}

In this section we describe our experiments on O-MCTS and OL-MCTS\@. The
algorithms are compared to the MCTS algorithm, as described in
Section~\ref{subsec:mcts}. All algorithms are run on a set of twenty-eight
different games in the VGDL framework. The set consists of all the games from
the first four training sets of the GVGAI competition, excluding puzzle games
that can be solved by an exhaustive search and have no random component (e.g.
NPCs). Each game has five levels.

Firstly, we compare O-MCTS to MCTS by showing the win ratio and mean score of
both algorithms on all the games. Secondly we show the improvement that OL-MCTS
makes compared to O-MCTS when it is allowed 4 games of learning time. 
%We demonstrate the progress it achieves by showing the first and last of the
%games it plays.  
Lastly we compare the three algorithms by summing up all the victories of all
the levels of each game.  
%Following the GVGAI competition's scoring method,
%algorithms are primarily judged on their ability to win the games. The scores
%they achieve are treated as a secondary objective.

For these experiments we construct an option set which is aimed at providing
action sequences for any type of game, since the aim here is general video game
playing.  Note that since the option set is a variable of the algorithm, either
a more specific or automatically generated set of options can be used by the
algorithm as well.

The set of option types consists of one option that executes a specific action
once, an option that avoids the nearest NPC by moving away from it, an option
that moves to a movable sprite until it is close to it (but not on it). There
are options that go to a movable sprite and options to go to a certain position
in the game.  Lastly we create an option that waits until an NPC is at a certain
distance and then fires the weapon. The specifics about the option set can be
found in the original thesis~\cite{de2016monte}.

For each option type, a subtype per visible sprite type is created during the
game. For each sprite, an option instance of its corresponding subtype is
created. For example, the game \textit{zelda} contains three different sprite
types (excluding the avatar and walls); monsters, a key and a portal. The first
level contains three monsters, one key and one portal. The aim of the game is to
collect the key and walk towards the portal without being killed by the
monsters. The score is increased by 1 if a monster is killed, i.e., its sprite
is on the same location as the sword sprite, if the key is picked up, or when
the game is won.  \emph{go to movable} and \emph{go near movable} options are
created for each of the three monsters and for the key. A \emph{go to position}
option is created for the portal.  One \emph{go to nearest sprite of type}
option is created per sprite type. One \emph{wait and shoot} option is created
for the monsters and one \emph{avoid nearest NPC} option is created. This set of
options is $O$, as defined in Section~\ref{subsec:options}. In a state where,
for example, all the monsters are dead, the possible option set $\mathbf{p}_s$
does not contain the \emph{avoid nearest NPC}, \emph{go to movable} and \emph{go
near movable} options for the monsters.

The \emph{go to \ldots} and \emph{go near \ldots} options utilize an adaptation
of the $A^*$ algorithm to plan their routes~\cite{hart1968formal}. An
adaptation is needed, because at the beginning of the game there is no knowledge
of which sprites are traversable by the avatar and which are not. Therefore,
during every move that is simulated by the agent, the $A^*$ module has to
update its beliefs about the location of walls and other blocking objects. This
is accomplished by comparing the movement the avatar wanted to make to the
movement that was actually made in game. If the avatar did not move, it is
assumed that all the sprites on the location the avatar should have arrived in
are blocking sprites. Our $A^*$ keeps a \emph{wall score} for each sprite type.
When a sprite blocks the avatar, its wall score is increased by one.
Additionally, to prevent the avatar from walking into deadly sprites, when a
sprite kills the avatar, its wall score is increased by 100.  Traditionally the
$A^*$'s heuristic uses the distance between two points. Our $A^*$ adaptation
adds the wall score of the goal location to this heuristic, encouraging the
algorithm to take paths with a low wall score. This method enables $A^*$ to
try to traverse paths that were unavailable earlier, while preferring safe and
easily traversable paths. For example in \textit{zelda}, a door is closed until
a key is picked up. Our $A^*$ implementation will still be able to plan a path
to the door once the key is picked up, to win the game. Note that because the
games can be stochastic, $A^*$ has to be recalculated for each simulation.

We empirically optimize the parameters of the algorithms for the experiments.
We use discount factor $\gamma = 0.9$, maximum action time $t = 40$
milliseconds. The maximum search depth $d$ is set to 70, which is higher than
most alternative tree search algorithms, for example in the GVGAI competition,
use. This is possible because the search tree has a relatively low branching
factor. The number of node visits after which \textsf{uct} is used, $v$, is set
to 40. Crazy stone parameter $K$ is set to $0.5$.  For comparison, we use the
MCTS algorithm provided with the Java implementation of VGDL which employs a
maximum tree depth of 10, because the branching factor is higher. Both
algorithms have \textsf{uct} constant $C_p = \sqrt{2}$. Unfortunately, comparing
to Q-learning with options was impossible, because the state space of these
games is too big for the algorithm to learn any reasonable behavior. All the
experiments are run on an Intel i7--2600, 3.40GHz quad core processor with 6 GB
of DDR3, 1333 MHz RAM memory. In all the following experiments on this game set,
each algorithm plays each of the 5 levels of every game 20 times.

%\subsection{O-MCTS}
%\label{subsec:omcts}
First, we will describe the results of the O-MCTS algorithm in comparison with
MCTS\@.  This demonstrates the improvement that can be achieved by using our
algorithm.  The games in this and the following
experiments are ordered from left to right by the performance of an algorithm
that always chooses a random action, indicating the complexity of the games.
Figure~\ref{fig:scores} shows the win ratio and normalized score of the
algorithms for each game. In short, the O-MCTS algorithms performs at least as
good as MCTS in almost all games, and better in more than half.

O-MCTS outperforms MCTS in the games \textit{missile command}, \textit{bait},
\textit{camel race}, \textit{survive zombies}, \textit{firestorms},
\textit{lemmings}, \textit{firecaster}, \textit{overload}, \textit{zelda},
\textit{chase}, \textit{boulderchase} and \textit{eggomania} winning more games
or achieving a higher mean score. By looking at the algorithm's actions for
these games, we can see that O-MCTS succeeds in efficiently planning paths in a
dangerous environment, enabling it to do a further forward search than MCTS\@. 
\textit{Camel race} requires the player to
move to the right for 80 consecutive turns to reach the finish line. No
intermediate rewards are given to indicate that the agent is walking in the
right direction. This is hard for MCTS, since it only looks 10 turns ahead.
O-MCTS always wins this game, since it can plan forward a lot further.
Furthermore, the rollouts of the option for walking towards the finish line have
a bigger chance of reaching the finish line than the random rollouts executed by
MCTS\@.
In \textit{zelda} we can see that the MCTS algorithm achieves roughly the same
score as O-MCTS, but does not win the game, since picking up the key and walking
towards the door is a difficult action sequence.  We assume that the mean score
achieved by MCTS is because it succeeds in killing the monsters, whereas O-MCTS
achieves its score by picking up the key and walking to the door.  These results
indicate that O-MCTS performs better than MCTS in games where a sequence
subgoals have to be reached.

The MCTS algorithm performs better than O-MCTS in \textit{pacman},
\textit{whackamole}, \textit{jaws}, \textit{seaquest} and \textit{plaque attack}
(note that for \textit{seaquest}, O-MCTS has a higher mean score, but wins less
than MCTS). A parallel between these games is that they have a big number of sprites, for each of which several options have to be
created by O-MCTS\@. When the number of options becomes too big, constructing the
set of possible options $\mathbf{p}_s$ for every state $s$ becomes so
time-consuming that the algorithm has too little time to build a tree and find
the best possible action. To test this hypothesis we ran the same test with an increased computation
time of 120ms and found that the win ratio of O-MCTS increases to
around $0.8$ for \textit{seaquest} and \textit{plaque attack}, whereas the win
ratio for MCTS increased to $0.9$ and $0.7$ respectively. This means that with
more action time, the difference between O-MCTS and MCTS is reduced for
\textit{seaquest} and O-MCTS outperforms MCTS on \textit{plaque attack}.

\begin{figure}
	\centering
	\includegraphics[width=\columnwidth]{includes/winsOLMCTS}
	%\vspace{-.8cm}
	%\caption{Win ratio comparison of OL-MCTS and O-MCTS.}
	%\label{fig:wins-olmcts}
	\centering
	\includegraphics[width=\columnwidth]{includes/scoresOLMCTS}
	\vspace{-.8cm}
	\caption{Normalized win ratio and score comparison of OL-MCTS and O-MCTS\@.
	OL-MCTS outperforms O-MCTS by a small margin in some games. In the games
	that are not shown both algorithms perform equally.}
\label{fig:scores-olmcts}
\end{figure}

%\subsection{OL-MCTS}
%\label{subsec:olmcts}
Secondly, we compare OL-MCTS to O-MCTS by running it on the same set of games.
The option learning algorithm is allowed four learning games, after which the
fifth is used for the comparisons. Figure~\ref{fig:scores-olmcts} shows the
performance difference between O-MCTS and OL-MCTS on some games. For the other
games, the performance was approximately the same. Here OL-MCTS1 shows the
performance of OL-MCTS on the first game. OL-MCTS5 shows the performance of the
algorithm after learning for four games. 

We can see that, although the first iteration of OL-MCTS sometimes performs a
bit worse than O-MCTS, the fifth iteration often scores at least as high, or
higher than O-MCTS\@. We expect that the loss of performance in OL-MCTS1 is
a result of the extra overhead that is added by the crazy stone algorithm: a
sorting of all the option values has to take place in each tree node. The
learning algorithm significantly improves score and win ratio for the game
\textit{bait}.
%, which is a game in which the objective is to reach a goal portal
%after collecting a key.  The player can push boxes around to open paths. There
%are holes in the ground that kill the player unless they are filled with boxes,
%which make both the hole and the box disappear.
Figure~\ref{fig:learning-results} shows the improvement in score and win ratio
for this game. There are two likely explanations for this improvement: 1. There
are sprites that kill the player, which the algorithm learns to avoid  2. The
algorithm learns that it should pick up the key.

Furthermore, we can see small improvements on the games \textit{seaquest},
\textit{plaque attack} and \textit{jaws}, on which O-MCTS performs worse than
MCTS\@.  Although OL-MCTS does not exceed the score of the MCTS algorithm, this
improvement suggests that OL-MCTS is on the right path of improving O-MCTS\@.

\begin{figure}
	\centering
	\subfigure[Learning \textit{bait}]{%
\includegraphics[scale=.44]{includes/learning}%
\label{fig:learning-results}%
	}
	\subfigure[Totals]{%
\includegraphics[scale=.44]{includes/totals.pdf}%
\label{fig:total-results}%
	}
	\caption{Learning improvement on game \textit{bait}, it shows win ratio and
		normalized score. Total number of wins of the algorithms on
	all games.}
\end{figure}

\subsection{Results}
\label{subsec:totals}
Summarizing, our tests indicate that on complex games O-MCTS outperforms MCTS\@.
For other games it performs at least as well, as long as the number of game
sprites is not too high.  The OL-MCTS algorithm can increase performance for
some of the games, such as \textit{bait} and \textit{plaque attack}. On other
games, little to no increased performance can be found.

An overview of the results is depicted in Figure~\ref{fig:total-results}, which
shows the sum of wins over all games, all levels.  It shows a significant ($p <
0.05$) improvement of O-MCTS and OL-MCTS over MCTS\@.  There is no significant
difference between performance of OL-MCTS over O-MCTS, although our results
suggest that it does improve for a subset of the games.
