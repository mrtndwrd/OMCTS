\section{Experiments}
\label{sec:experiments}
\begin{figure*}
	\centering
	\includegraphics[width=\textwidth]{includes/wins}
	\vspace{-.8cm}
	\caption{Win ratio of the algorithms per game on all levels.}
	\label{fig:wins}
\end{figure*}

\begin{figure*}
	\centering
	\includegraphics[width=\textwidth]{includes/scores}
	\vspace{-.8cm}
	\caption{Mean normalized score of the algorithms per game 1 means the
	highest score achieved by all the algoriths, 0 the lowest.}
	\label{fig:scores}
\end{figure*}

In order to test if O-MCTS improves upon MCTS, and if OL-MCTS improves upon
O-MCTS, this section simulates the algorithms on 28 different games in the VGDL
framework. These games include all games from the first four training sets of
the GVGAI competition, excluding puzzle games that can be solved by a simple
exhaustive search and have no random component (like NPCs, for example). Each
game consists of 5 levels. 

For this experiment we construct an option set which is aimed at providing
action sequences for any type of game, since the aim here is general video game
playing. Note that a more specific set of options can be created when the
algorithm should be tailored to only one type of games and similarly, more
options can be added to the algorithm easily.

\begin{itemize}[noitemsep]
	\item \texttt{ActionOption} executes a specific action once and then
		stops.
	\item \texttt{AvoidNearestNpcOption} makes the agent avoid the nearest NPC
	\item \texttt{GoNearMovableOption} makes the agent walk towards a
		movable game sprite (defined as movable by the VGDL) and stops when it
		is within a certain range of the movable
	\item \texttt{GoToMovableOption} makes the agent walk towards a
		movable until its location is the same as that of the movable
	\item \texttt{GoToNearestSpriteOfType} makes the agent walk to the nearest sprite of
		a specific type
	\item \texttt{GoToPositionOption} makes the agent walk to a specific position
	\item \texttt{WaitAndShootOption} waits until an NPC is in a specific location and
		then uses its weapon.
\end{itemize}

For each option type, a subtype per visible sprite type is created during the
game. For each sprite, an option instance of its corresponding subtype is
created. For example, the game \textit{zelda}, as seen in Figure \ref{fig:zelda},
contains three different sprite types (excluding the avatar and walls);
monsters, a key and a portal. The first level contains three monsters, one key
and one portal, and the aim of the game is to collect the key and walk towards
the portal without walking into the monsters. The score is increased by 1 if a
monster is killed (i.e., its sprite is on the same location as the sword sprite)
if the key is picked up, and when the game is won. \texttt{GoToMovableOption} and
\texttt{GoNearMovableOptions} are created for each of the three monsters and
for the key. A \texttt{GoToPositionOption} is created for the portal.  One
\texttt{GoToNearestSpriteOfType} is created per sprite type. One
\texttt{WaitAndShootOption} is created for the monsters, and one
\texttt{AvoidNearestNpcOption} is created. This set of options is $O$ in
Algorithms \ref{alg:omcts} and \ref{alg:olmcts}. In a state where, for example,
all the monsters are dead, the possible option set $\mathbf{p}_s$ does not
contain the \texttt{AvoidNearestNpcOption} and \texttt{GoToMovableOption}s and
\texttt{GoNearMovableOption}s for the monsters.

The \texttt{GoTo\ldots} options all utilize an adaptation of the A Star
algorithm to plan routes. An adaptation is needed, because at the beginning of
the game there is no knowledge of which sprites are traversable and which are
not. Thus, during every move that is simulated by the agent, the A Star module
has to update its beliefs about the location of walls and other blocking
objects. This is accomplished by comparing the movement the avatar wanted to
make to the movement that was actually made in game. If the avatar did not move,
it is assumed that all the sprites on the location the avatar should have
arrived in are blocking sprites. A Star keeps a \emph{wall score} for each
sprite type. When a sprite blocks the avatar, its wall score is increased by
one. Additionally, when a sprite kills the avatar, its wall score is increased
by 100, in order to prevent the avatar from walking into killing sprites.
Traditionally the A Star's heuristic uses the distance between two points. Our A
Star adaptation adds the wall score of the goal location to this heuristic,
encouraging the algorithm to take paths with a lower wall score. This method
enables A Star to try to traverse paths that were unavailable earlier, while
preferring safe and easily traversable paths. For example in \textit{zelda}, a
door is closed until a key is picked up. Our A Star version will still be able
to plan a path to the door once the key is picked up, winning the game.

We empirically optimize the parameters of the O(L)-MCTS algorithm
for these experiments. We use discount factor $\gamma = 0.9$, maximum action
time $t = 40$ milliseconds. The maximum search depth $d$ is set to 70, which is
higher than most alternative tree search algorithms, for example in the GVGAI
competition, use. The number of node visits after which \textsf{uct} is used,
$v$, is set to 40. Crazy stone parameter $K$ is set to $0.5$.  For comparison,
we use the MCTS algorithm provided with the Java implementation of VGDL with its
default value of maximum search depth of 8. Both algorithms have \textsf{uct}
constant $C_p = \sqrt{2}$. Unfortunately, comparing to Q-learning with options
was impossible, because the state space of these games is too big for the
algorithm to learn any reasonable behavior. All the experiments are run on an
Intel i7 \todo{type} quad core processor with 6 GB of RAM memory.

In order to compare the algorithms, each algorithm plays each of the 5 levels
of each game 20 times. The goal of the algorithms is to win as many games as
possible, while maximizing the score. The transfer learning algorithm is
allowed four learning games, after which the fifth is used for the
comparisons. Figures \ref{fig:wins} and \ref{fig:scores} respectively show the
win ratio and normalized score for each game and algorithm. The games are
ordered by the performance of a random algorithm, indicating the complexity of
the games. From left to right, the random algorithm's win ratio and score
decreases. As can be seen in Figure \ref{fig:wins}, the O-MCTS algorithms
performs at least as good as MCTS in almost all games. The MCTS algorithm
performs significantly better than O-MCTS in \textit{whackamole}, \textit{jaws},
\textit{seaquest} and \textit{plaque attack}. The similarity of these games is
that they have a very big number of different sprites, for each of which several
options have to be created by O-MCTS.  When the number of options becomes too
big, constructing the set of possible options $\mathbf{p}_s$ for every state $s$
becomes so time-consuming that the algorithm has too little time to build a tree
and find the best possible action. When the computation time is increased to
120ms, the win ratio of O-MCTS increases to around $0.8$ for \textit{seaquest}
and \textit{plaque attack}.

Contrarily, O-MCTS greatly outperforms MCTS in the games \textit{missile
command}, \textit{overload}, \textit{firestorms}, \textit{boulderchase},
\textit{zelda}, \textit{bait}, \textit{camel race}, \textit{eggomania},
\textit{firecaster}, \textit{chase} and \textit{lemmings}, winning many more
games. Analyzing the algorithm's actions for these games, we can see that
the algorithm succeeds in efficiently planning paths in a dangerous environment,
enabling it to do a further forward search than the ordinary Monte Carlo tree
search. \textit{Camel race} requires the player to move to the right for 80
consecutive turns, which is very hard for MCTS, since it only looks forward 8
turns. O-MCTS almost always wins, since it can plan forward a lot further. In
Overload, a sword has to be picked up before the avatar can finish the game,
which seems to be too hard for MCTS, but poses less of a problem for O-MCTS.
Furthermore, in Zelda we can see that the MCTS algorithm achieves roughly the
same score as O-MCTS, but does not win the game, since picking up the key and
walking towards the door is a difficult action sequence. We assume that the
score achieved by MCTS is because it succeeds in killing the monsters, whereas
O-MCTS achieves its score by picking up the key and walking to the door. We can
conclude that O-MCTS and OL-MCTS perform better than MCTS in games where a
longer sequence of action planning has to be done.

\begin{figure}
	\centering
	\subfigure[Learning \textit{bait}]{
		\includegraphics[scale=.44]{includes/learning}
		\label{fig:learning-results}
	}
	\subfigure[Totals]{
		\includegraphics[scale=.44]{includes/totals.pdf}
		\label{fig:total-results}
	}
	\caption{Learning improvement on game \textit{bait}, it shows win ratio and
		normalized score. Total number of wins of the algorithms on
	all games.}
\end{figure}

In order to test the improvement of the transfer learning algorithm, we compare
the blue bars with the red. We can see that the red bar is at least as high as
the blue bar in the majority of the games. Furthermore, the transfer learning
algorithm significantly improves score and win ratio for the game \textit{bait},
which is a game in which the objective is to reach a goal portal after
collecting a key. The player can push boxes around to open paths. There are
holes in the ground that kill the player, unless they are filled with boxes,
which make both the hole and the box disappear. Figure
\ref{fig:learning-results} shows the improvement in score and win ratio for this
game. There are three possible reasons for this improvement: 1.) There are
sprites that kill the player, which will be evaded by the algorithm when it has
learned to do so.  2.) The algorithm might learn that it should pick up the key.
3.) The player might learn to apply options in such a way that he can push boxes
into the (deadly) holes.

Figure \ref{fig:total-results} shows the sum of wins over all games, all levels.
It shows a significant ($p < 0.05$) improvement of O-MCTS, OL-MCTS and OTL-MCTS
over MCTS. OL-MCTS also slightly, but significantly improves on itself over the
course of 5 plays of these games, although it does not outperform O-MCTS in its
first game-play.

Summarizing, our tests indicate that on complex games O-MCTS outperforms MCTS.
For other games it performs at least as well, as long as the number of game
sprites is not too high. The OL-MCTS algorithm performs worse than O-MCTS before
learning but can, in specific situations, increase performance significantly.
