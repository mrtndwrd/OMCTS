\section{Learning}
\label{sec:learning}
Although O-MCTS might improve the tree search, when the number of options
increases, the branching factor of O-MCTS increases as well. When many options
are defined exploring all the options, for each node where an option is
finished, becomes infeasible. In this section, we will define \emph{option
values}, the expected mean and variance of an option, which can be used to
estimate which options are feasible and which are not. Exploration can then
focus on the feasible options, enabling an even deeper search tree than O-MCTS,
in the same time. We expect that this increases performance, especially in games
where the set of possible options is large, or where only a small set of options
is needed in order to win.

The return $R_o$ for using option $o$ from timestep $t$ to $t+n$ is calculated
by adding the discounted rewards $r_t$ for all of the states visited by that
option \cite{sutton1999between}. $$R_o = r_{t} + \gamma r_{t+1} + \gamma^2 r_{t+2} + \cdots + \gamma^n
r_{t+n},$$ where $\gamma \in [0, 1]$ is the discount rate parameter, which
influences the importance of rewards that lay further in the future: an option
with reward 1 at timestep $t$ will get a greater return than an option with the
same reward at timestep $t+1$.  

The set of options consists of different options of the same \emph{type}. For example,
there is an option for going to a movable sprite, which has type
\texttt{GoToMovableOption}. An instance of this option exists for each movable
sprite in the game. \emph{Subtypes} are defined as well. In this case, each
unique game sprite has its own subtype. The mean and variance of the options'
discounted rewards are saved and calculated per subtype. Each time an option is
finished, its subtype's values $\mu_o$ and $\sigma_o$ are updated by
respectively taking the mean and variance of all the returns of this subtype. By
saving values per subtype the algorithm can generalize over sprites of the same
type.

Because we have an option's expected mean and variance we can use the crazy
stone algorithm from Equation \ref{eq:crazystone} to reduce calculation time by
shifting the focus of the exploration to promising options. The crazy stone
algorithm is applied in the expansion phase of O-MCTS. As a result, not all
children of a node will be expanded, but only the ones selected by crazy stone.
Because crazy stone can select the same option several times, it enables deeper
exploration of promising subtrees. 

Crazy stone uses the mean return $R_o$ of each option as its estimated value
$\mu$ for choosing which option to select or expand. After a predefined number
of visits $v$ to a node, the selection strategy \textsf{uct} is followed to
tweak the option selection. When it starts using \textsf{uct}, no new expansions
are done in this node for the remainder of time.

The new algorithm can be seen in Algorithm \ref{alg:olmcts}. The updates of the
option values are done in line \ref{alg:olmcts:update}. The function
\textsf{update\_values} takes the return of the option that was used, and
updates its $\mu$ and $\sigma$ by calculating the new mean and variance of all
returns. The next difference with O-MCTS starts in line \ref{alg:olmcts:ns},
where the algorithm applies crazy stone if the current node has been visited
less than $v$ times, or alternatively applies \textsf{uct} like in O-MCTS.
\textsf{crazy\_stone} returns a set of weights over options. A weighted random
then chooses a new option $\omega$ by using these weights.  If $\omega$ has not
been explored yet, i.e., there is no child node of $s$ in $c_s$ that uses this
option, the algorithm chooses and applies an action, and breaks to rollout in
lines \ref{alg:olmcts:scs} to \ref{alg:olmcts:ecs}. This is similar to the
exploration steps in O-MCTS. If $\omega$ has been explored in this node before,
the corresponding child node $s'$ is selected from $c_s$ and the loop continues
similar to when \textsf{uct} selects a child.

The information learned in a game can be transferred if the game is played
again by supplying OL-MCTS with the $\mu$ and $\sigma$ of the previous game. We
will refer to this as \emph{Option Transfer Learning MCTS (OTL-MCTS)}. Especially
in games where some options excel over the others, transfer learning can help
identifying these and focussing exploration on these options. 

We expect that by learning option values and applying crazy stone, the algorithm
can create even deeper search trees, focussed more on promising areas of the
search space, thereby improving its performance more than O-MCTS. Furthermore,
we expect that transferring option values to the next game, the algorithm can
improve over the course of several games.

\begin{algorithm}
	\caption{$\mathsf{OL-MCTS}(O, r, t, d, v, \mu, \sigma)$}
	\label{alg:olmcts}
	\begin{algorithmic}[1]
		\State $C_{s \in S} \gets \emptyset$
		\State $\mathbf{o} \gets \emptyset$
		\While {$time\_taken < t$} \label{alg:olmcts:mainloop}
			\State $s \gets r$
			\While {$\neg \mathsf{stop}(s, d)$} \label{alg:olmcts:innerloop}
				\If{$s \in \beta(o_s)$} \label{alg:olmcts:sp}
					\State $\mathsf{update\_values}(s, o_s, \mu, \sigma)$
						\Comment{Update $\mu$ and $\sigma$} \label{alg:olmcts:update}
					\State $\mathbf{p}_s \gets \cup_o (s \in I_{o \in O})$
				\Else
					\State $\mathbf{p}_s \gets \{o_s\}$
				\EndIf \label{alg:olmcts:scs}
				\State $\mathbf{m} \gets \cup_o (o_{s \in \mathbf{c}_s})$
				\If{$n_s < v$} \Comment{Apply \textsf{crazy\_stone}} \Comment{Eq. \ref{eq:crazystone}}
					\label{alg:olmcts:ns}
					\State $\mathbf{u}_s \gets \mathsf{crazy\_stone}(\mu, \sigma, \mathbf{p}_s)$
					\State $\omega \gets \mathsf{weighted\_random}(\mathbf{u}_s, \mathbf{p}_s)$
					\If{$\omega \not\in \mathbf{m}$} \Comment{option $\omega$ not expanded}
						\State $a \gets \mathsf{get\_action}(\omega, s)$ \label{alg:olmcts:scs}
						\State $s' \gets \mathsf{expand}(s, a)$ 
						\State $\mathbf{c}_s \gets \mathbf{c}_s \cup \{s'\}$
						\State $o_{s'} \gets \omega$
						\State \textbf{break} \label{alg:olmcts:ecs}
					\Else
						\State $s' \gets s \in \mathbf{c}_s : o_s = \omega$ \label{alg:olmcts:s}
					\EndIf
				\Else \Comment{Apply \textsf{uct}}
					\State $s' \gets \mathsf{uct}(s)$ \label{alg:olmcts:uct}
				\EndIf \label{alg:olmcts:ecs}
				\State $s \gets s'$ \label{alg:olmcts:ss}
			\EndWhile
			\State $\delta \gets \mathsf{rollout}(s')$ \label{alg:olmcts:rollout}
			\State $\mathsf{back\_up}(s', \delta)$ \label{alg:olmcts:backup}
		\EndWhile
	\end{algorithmic}
\end{algorithm}
